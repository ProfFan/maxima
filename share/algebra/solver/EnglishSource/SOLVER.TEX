\begin{verbatim}
/******************************************************************************/
/*                                                                            */
/* SOLVER - THE NEXT GENERATION                                               */
/*                                                                            */
/* Copyright (C) 2000 : Eckhard Hennig, Ralf Sommer                           */
/* This library is free software; you can redistribute it and/or modify it    */
/* under the terms of the GNU Library General Public License as published     */
/* by the Free Software Foundation; either version 2 of the License, or (at   */
/* your option) any later version.                                            */
/*                                                                            */
/* This library is distributed in the hope that it will be useful, but        */
/* WITHOUT any WARRANTY; without even the implied warranty of                 */
/* MERCHANTABILITY or FITNESS for A PARTICULAR PURPOSE.  See the GNU          */
/* Library General Public License for more details.                           */
/*                                                                            */
/* You should have received a copy of the GNU Library General Public          */
/* License along with this library; if not, write to the Free Software        */
/* Foundation, Inc., 51 Franklin Street, Fifth Floor, Boston, MA  02110-1301, USA                  */
/******************************************************************************/
/* Credits: This program is based on many ideas from Henning Trispel's work   */
/*          on the EASY-Solver in 1991.                                       */
/*                                                                            */
/******************************************************************************/
/* Author(s)    : Eckhard Hennig, Ralf Sommer                                 */
/* Project start: 19.01.1994                                                  */
/* Completed    : 16.07.1994                                                  */
/* last change  : 29.06.1995                                                  */
/* Time         : 09:26                                                       */
/******************************************************************************/
/* Changes      : ||||| ||||| ||||| |||                                       */
/******************************************************************************/
/* Modified by  : Dan Stanger dan.stanger@ieee.org to work under maxima       */
/******************************************************************************/
load( "solver/linsolve.mac" )$
load( "solver/slvrtbox.mac" )$
load( "solver/slvrmsgs.mac" )$
load( "solver/misc.mac" )$

put( 'SOLVER, 1, 'Version )$

SetVersion(
  /* KEY    = */ 'SOLVER,
  'MODULE      = "SOLVER",
  'DESCRIPTION = "Symbolic solver for parametric systems of equations.",
  'AUTHORS     = "Eckhard Hennig, Ralf Sommer",
  'DATE        = "19.01.1994",
  'LASTCHANGE  = "29.06.1995",
  'TIME        = "09:26",
  'PLAN        = "Add a-priori transforms"
)$


/******************************************************************************/
/* last change: 29.06.1995                                                    */
/* Time       : 09:26                                                         */
/* By         : Eckhard Hennig                                                */
/* Description: Option variable Solver_Valuate_All_Nonlin_Vars added.         */
/*              Bug in ValuationSolver removed: rhs variables of solutions    */
/*              were not added to the list of variables.                      */
/******************************************************************************/
/* last change: 28.05.1995                                                    */
/* Time       : 11:59                                                         */
/* By         : Eckhard Hennig                                                */
/* Description: Solver break test added.                                      */
/******************************************************************************/
/* last change: 18.05.1995                                                    */
/* Time       : 16:10                                                         */
/* By         : Eckhard Hennig                                                */
/* Description: Name conflict with AI keyword PARAMS removed.                 */
/******************************************************************************/
/* last change: 30.01.1995                                                    */
/* Time       : 21.42                                                         */
/* By         : Eckhard Hennig                                                */
/* Description: Removal of map( 'num, ... ) has revealed some side effects on */
/*              the list of equations in ValuationSolver. Bug repaired by     */
/*              inserting two additional copylist calls.                      */
/*              Arguments to both load commands above now written in lower-   */
/*              case letters to avoid problems with UNIX versions.            */
/******************************************************************************/
/* last change: 24.01.1995                                                    */
/* Time       : 16.33                                                         */
/* By         : Eckhard Hennig                                                */
/* Description: mapping 'num to expressions caused some invalid solutions,    */
/*              therefore removed. Version property added.                    */
/******************************************************************************/
/* last change: 19.01.1995                                                    */
/* Time       : 13.04                                                         */
/* By         : Eckhard Hennig, Ralf Sommer                                   */
/* Description: Bug in ImmediateAssignments corrected, Function AppendImmed   */
/*              removed.                                                      */
/******************************************************************************/
/* last change: 19.01.1995                                                    */
/* Time       : 11.28                                                         */
/* By         : Eckhard Hennig, Ralf Sommer                                   */
/* Description: SLVRTBOX and SLVRMSGS now autoloading.                        */
/******************************************************************************/
/* last change: 17.01.1995                                                    */
/* Time       : 20.15                                                         */
/* By         : Eckhard Hennig, Ralf Sommer                                   */
/* Description: Option variables modified (underscores inserted according to  */
/*              R. Petti's suggestions)                                       */
/*              Function SetValuation added.                                  */
/******************************************************************************/
/* last change: 25.10.1994                                                    */
/* Time       : 12.46                                                         */
/* By         : Eckhard Hennig                                                */
/* Description: errorMsg renamed to ErrMsg.                                   */
/******************************************************************************/
/* last change: 12.09.1994                                                    */
/* Time       : 11.50                                                         */
/* By         : Eckhard Hennig                                                */
/* Description: errcatch wrapped around final fullratsimp.                    */
/******************************************************************************/
/* last change: 05.09.1994                                                    */
/* Time       : 15.59                                                         */
/* By         : Eckhard Hennig                                                */
/* Description: mode_identity's inserted.                                     */
/******************************************************************************/
/* last change: 05.09.1994                                                    */
/* Time       : 15.21                                                         */
/* By         : Eckhard Hennig                                                */
/* Description: Number of linear equations is now correctly displayed.        */
/*              Bug in DumpToFile corrected.                                  */
/*              Numbers of solution sets are now printed by the postprocessor.*/
/******************************************************************************/
/* last change: 30.08.1994                                                    */
/* Time       : 13.45                                                         */
/* By         : Eckhard Hennig                                                */
/* Description: Single inconsistent solution paths are no longer returned.    */
/******************************************************************************/
/* last change: 29.08.1994                                                    */
/* Time       : 17.16                                                         */
/* By         : Eckhard Hennig                                                */
/* Description: Slight modification of Immediate Assignment Solver.           */
/*              Valuation property for sqrt added.                            */
/*              New option variable SolverDefaultValuation.                   */
/*              Change in Valuation. Now making use of the above option var.  */
/******************************************************************************/
/* last change: 26.08.1994                                                    */
/* Time       : 15.29                                                         */
/* By         : Eckhard Hennig                                                */
/* Description: Bug in Linear Solver removed.                                 */
/******************************************************************************/
/* last change: 25.08.1994                                                    */
/* Time       : 12.41                                                         */
/* By         : Eckhard Hennig                                                */
/* Description: New option variable SolverRatSimpSols.                        */
/*              Linear Solver now calls the consistency check.                */
/*              Capabilities of the transforms in ValuationSolver strongly    */
/*              enhanced.                                                     */
/******************************************************************************/
/* last change: 24.08.1994                                                    */
/* Time       : 23.14                                                         */
/* By         : Eckhard Hennig                                                */
/* Description: Bug fixed in postprocessor: compound expressions are now cor- */
/*              rectly evaluated.                                             */
/*              Linear Solver now checks for remaining equations & variables, */
/*              and removes those "linear" eqs and vars which do not really   */
/*              belong to the "true" linear subsystem.                        */
/******************************************************************************/
/* last change: 11.08.1994                                                    */
/* Time       : 19.09                                                         */
/* By         : Eckhard Hennig                                                */
/* Description: Variable SolverDelEq2VarPref replaced by function variable    */
/*              SolverDelEq. Heuristic algorithm for linear equation          */
/*              extraction improved.                                          */
/******************************************************************************/
/* last change: 19.07.1994                                                    */
/* Time       : 18.01                                                         */
/* By         : Eckhard Hennig                                                */
/* Description: Bug in post processor corrected.                              */
/******************************************************************************/


/******************************************************************************/
/* Global variables for Solver                                                */
/******************************************************************************/

/******************************************************************************/
/* If Solver_Immed_Assign is true then the Solver searches the equations      */
/* for immediate assignments of the form variable = constant and immediately  */
/* inserts these constraints into the remaining equations.                    */
/******************************************************************************/

define_variable( Solver_Immed_Assign, true, boolean )$


/******************************************************************************/
/* Solver_Repeat_Immed controls whether the search for immediate assignments  */
/* is performed repeatedly until no more of them are found.                   */
/******************************************************************************/

define_variable( Solver_Repeat_Immed, true, boolean )$


/******************************************************************************/
/* Solver_Subst_Powers controls whether the Solver substitutes powers of a    */
/* variable by new symbols in one of the following cases:                     */
/* 1. var^n appears raised to exactly one power n                             */
/* 2. for all var^m in the equations : m=n*k , k integer                      */
/******************************************************************************/

define_variable( Solver_Subst_Powers, false, boolean )$


/******************************************************************************/
/* Solver_Incons_Params controls whether the Solver terminates when a non-    */
/* trivial equation containing only parameters is encountered. for example,   */
/* if A and B are defined as parameters and the solution process yields       */
/* A = B^2  then the Solver stops if Solver_Incons_Params is = 'BREAK. If set */
/* to 'ASK the Solver asks whether  A - B^2  is zero and continues if it is.  */
/* If set to 'IGNORE the Solver quietly assumes that the expression is zero   */
/* if it does not directly contradict with any of the assumptions made before.*/
/******************************************************************************/

define_variable( Solver_Incons_Params, 'ASK, any_check )$

put(
  'Solver_Incons_Params,
  lambda( [ x ],
    if not member( x, [ 'ASK, 'BREAK, 'IGNORE ] ) then
      ErrorHandler("InvIncPar", x, 'Fatal )
  ),
  'value_check
)$


/******************************************************************************/
/* Solver_Linear controls whether the Solver tries to find and solve blocks   */
/* of linear equations before submitting the remaining equations to the       */
/* heuristic valuation solver. This is useful if the equations to be solved   */
/* are known to contain a large linear part.                                  */
/******************************************************************************/

define_variable( Solver_Linear, true, boolean )$


/******************************************************************************/
/* If Solver_Repeat_Linear is true then the LinearSolver will be called       */
/* repeatedly until no more linear equations are found. If false then         */
/* LinearSolver will be called only once.                                     */
/******************************************************************************/

define_variable( Solver_Repeat_Linear, true, boolean )$


/******************************************************************************/
/* If Solver_Find_All_Linear_Vars is true then LinearSolver will try to find  */
/* linear equations with respect to all available variables and solve these   */
/* equations simultaneously. If false then LinearSolver will only search for  */
/* linear equations with respect to the variables passed over in the function */
/* call.                                                                      */
/******************************************************************************/

define_variable( Solver_Find_All_Linear_Vars, true, boolean )$


/******************************************************************************/
/* Solver_Assumptions contains constraints on the parameters which should be  */
/* checked by the user after the termination of Solver. These constraints     */
/* result from the parameter consistency check the behavior of which is       */
/* controlled by the setting of the option variable Solver_Incons_Params.     */
/* Any numerical solution of the equations obtained by assigning numerical    */
/* values to symbolic parameters should be checked for consistency with all   */
/* expressions in Solver_Assumptions.                                         */
/******************************************************************************/

define_variable( Solver_Assumptions, [], list )$


/******************************************************************************/
/* Solver_Del_Eq holds the name of a function which controls the behavior of  */
/* the heuristic search algorithm which extract linear equations from the     */
/* entire system of equations.                                                */
/******************************************************************************/

define_variable( Solver_Del_Eq, 'MakeSquareLinearBlocks, any )$


/******************************************************************************/
/* If Solver_Valuate_All_Nonlin_Vars is true then the ValuationSolver will    */
/* valuate the equations w.r.t. all remaining variables and not only w.r.t    */
/* variables which are being searched for at the current step.                */
/******************************************************************************/

define_variable( Solver_Valuate_All_Nonlin_Vars, false, boolean )$


/******************************************************************************/
/* Solver_Valuation_Strategy holds the name of the equation valuation         */
/* strategy called by the valuation solver to determine the order by which    */
/* the equations are to be solved.                                            */
/******************************************************************************/

define_variable( Solver_Valuation_Strategy, 'MinVarPathsFirst, any_check )$

put(
  'Solver_Valuation_Strategy,
  lambda(
    [ x ],
    if not FunctionP( x ) then
      ErrorHandler( "UndefStrat", x, 'Fatal )
  ),
  'value_check
)$


/******************************************************************************/
/* Solver_Default_Valuation contains the default valuation for arithmetic     */
/* operators. Whenever an operator is encountered for which no valuation has  */
/* been defined by SetProp( <operator>, 'Valuation, <valuation> ) this value  */
/* is taken for the formula complexity calculations.                          */
/******************************************************************************/

define_variable( Solver_Default_Valuation, 10, fixnum )$


/******************************************************************************/
/* Solver_Max_Len_Val_Order limits the length of the list of candidates for   */
/* the solve calls in ValuationSolver. A low value will usually increase the  */
/* efficiency of the valuation solver since, in general, the first or second  */
/* attempt to solve an equation (hopefully) succeeds.                         */
/******************************************************************************/

define_variable( Solver_Max_Len_Val_Order, 5, fixnum )$


/******************************************************************************/
/* Solver_Transforms is a list containing the names of functions which can be */
/* applied to an equation after a failed solve call. These functions must     */
/* take three arguments: the equation to be transformed, the variable to be   */
/* solved for, and a list of (probably implicit) solutions the Solver has     */
/* already found for the equation.                                            */
/******************************************************************************/

define_variable( Solver_Transforms, [], list )$


/******************************************************************************/
/* If Solver_Postprocess is set to false no postprocessing of the results will*/
/* be done. Instead, the solutions are displayed in the internal hierarchical */
/* list format. Useful for debugging purposes.                                */
/******************************************************************************/

define_variable( Solver_Postprocess, true, boolean )$


/******************************************************************************/
/* Solver_Backsubst controls the output format of Solver. If Solver_Backsubst */
/* is true then the result will be displayed with fully evaluated right-hand  */
/* sides for each variable. If the option variable is set to false then the   */
/* right-hand sides of the solutions may still contain references to some     */
/* of the other variables which have been solved for.                         */
/******************************************************************************/

define_variable( Solver_Backsubst, true, boolean )$


/******************************************************************************/
/* With Solver_Disp_All_Sols set to true all solutions will be displayed      */
/* including those for the variables which have been solved for in the        */
/* solution process but have not been explicitly asked for.                   */
/******************************************************************************/

define_variable( Solver_Disp_All_Sols, false, boolean )$


/******************************************************************************/
/* If Solver_RatSimp_Sols is true then the Solver Postprocessor will          */
/* fullratsimp the solutions before returning them.                           */
/******************************************************************************/

define_variable( Solver_RatSimp_Sols, true, boolean )$


/******************************************************************************/
/* If Solver_Dump_To_File is true then the ValuationSolver writes the         */
/* solutions and yet unsolved equations to a file after each iteration. This  */
/* might help if the Solver crashes.                                          */
/******************************************************************************/

define_variable( Solver_Dump_To_File, false, boolean )$


/******************************************************************************/
/* Solver_Dump_File contains the name of the file to which the dump is        */
/* written.                                                                   */
/******************************************************************************/

define_variable( Solver_Dump_File, "SOLVER.DMP", any )$


/******************************************************************************/
/* Solver_Break_Test holds the name of a function which is called immediately */
/* before attempting to solve an equation. This function then decides whether */
/* Solver should try to solve the equation or whether it should stop, e.g.    */
/* because the problem has become too complex. The arguments passed to the    */
/* Solver_Break_Test are 1. the equation, 2. the variable, 3. the valuation.  */
/* The Solver stops if the Solver_Break_Test returns true, it continues if    */
/* the return value is false.                                                 */
/******************************************************************************/

define_variable( Solver_Break_Test, 'SolverJustDoIt, any )$


/******************************************************************************/
/* Solver, main program                                                       */
/******************************************************************************/

Solver( Equations, [ SolverParams ] ) := (

  mode_declare(
    [ Equations, SolverParams ], list
  ),

  block(

    [
      Variables,       /* List of variables to be solved for            */
      UserVars,        /* List of variables specified by user           */
      Parameters,      /* List of symbols to be used as parameters      */
      Expressions,     /* List of compound expressions to be solved for */
      PowerSubst,      /* List of substituted symbols for powers        */

      Solutions,       /* List of solutions found by Solver             */
      RemainingEqs,    /* List of remaining equations                   */

      /* Assumptions made in linear solver should be local  */
      Solver_Assumptions : [],

      Active
    ],

    mode_declare(
      [
        Variables, UserVars, Parameters, Expressions, PowerSubst,
        Solutions
      ], list,
      Active, boolean
    ),

    /* No assumptions to start with */ErrorHandlerSolver_Assumptions : [],

    /* Initialize list of solutions */
    Solutions : [],

    /* Do all necessary preprocessing */
    map(
      lambda([x,y],x::y),
      [
        'Equations, 'SolverParams, 'Variables, 'Parameters,
        'Expressions, 'PowerSubst, 'UserVars
      ],
      SetupSolver( Equations, SolverParams )
    ),

    if MsgLevel = 'DEBUG then
      display(
        Equations, SolverParams, Variables, Parameters, Expressions,
        PowerSubst, UserVars, Solutions
      ),

    block(
      [],

      /* Search for and apply immediate assignments */

      Active : Solver_Immed_Assign,
      while Active do (
        map(
          lambda([x,y],x::y),
          [ 'Active, 'Solutions, 'Equations, 'Variables ],
          ImmediateAssignments( Solutions, Equations, Variables, Parameters )
        ),

        Active : Active and Solver_Repeat_Immed,

        if MsgLevel = 'DEBUG then
          display(
            Equations, SolverParams, Variables, Parameters, Expressions,
            PowerSubst, UserVars, Solutions
          )
      ),

      if Empty( Equations ) or Empty( Variables ) then
        return( false ),

      Equations : map(
        lambda(
          [ Eq ],
          fullratsimp( lhs( Eq ) - rhs( Eq ) )
        ),
        Equations
      ),

      /* Find and solve linear equations */

      Active : Solver_Linear,
      while Active do (
        map(
          lambda([x,y],x::y),
          [ 'Active, 'Solutions, 'Equations, 'Variables ],
          LinearSolver( Solutions, Equations, Variables, Parameters )
        ),

        Active : Active and Solver_Repeat_Linear,

        if MsgLevel = 'DEBUG then
          display(
            Equations, SolverParams, Variables, Parameters, Expressions,
            PowerSubst, UserVars, Solutions
          )
      ),

      if Empty( Equations ) or Empty( Variables ) then
        return( false ),


      if Solver_Valuate_All_Nonlin_Vars then
        Variables : Union(
          SetDifference( listofvars( Equations ), Parameters ),
          Variables
        ),

      /* apply valuation strategies to solve the nonlinear equations. */
      map(
        lambda([x,y],x::y),
        [ 'Active, 'Solutions, 'Equations, 'Variables ],
        ValuationSolver( Solutions, Equations, Variables, Parameters )
      )

    ), /* END block */


    /* Return the solutions and the unsolved equations */

    if Solver_Postprocess then
      return( PostProcess( Solutions, UserVars, Expressions, PowerSubst ) )
    else
      return( Solutions )
  )
)$


/******************************************************************************/
/* TerminateSolver terminates the Solver.                                     */
/******************************************************************************/

TerminateSolver() := error( ErrMsg["SolvrTerm"] )$


/******************************************************************************/
/* SetupSolver preprocesses the equations and optional parameters before      */
/* submitting them to the Solver. The equations are checked if they are       */
/* really equations, and all equations of the form  NUMBER = NUMBER  or       */
/* f( PARAMETERS ) = g( PARAMETERS )  are checked for consistency and then    */
/* dropped.                                                                   */
/******************************************************************************/

SetupSolver( Equations, SolverParams ) := (

  mode_declare(
    [ Equations, SolverParams ], list
  ),

  block(

    [
       i, AllVars, Var, SubstSym, Power,
       Variables, Parameters, Expressions,
       PowerSubst, UserVars
    ],

    mode_declare(
      i, fixnum,
      [
        AllVars, Power, Variables, Parameters, Expressions,
        PowerSubst, UserVars
      ], list,
      [ Var, SubstSym ], any
    ),


    Expressions : [],
    PowerSubst  : [],
    Parameters  : [],


    /* Make sure that Equations is a list. Abort if it is not. */

    if not listp( Equations ) then
      ErrorHandler( "EqsNotLst", Equations, 'Fatal ),

    /* delete all entries from Equations which are not equations */

    Equations : sublist( Equations, 'EquationP ),

    /* Convert all floating-point numbers to rational numbers. If this was */
    /* not done then rounding errors could fool the consistency check.     */

    Equations : map( 'rat, Equations ),

    /* Process the optional arguments to Solver */

    if not Empty( SolverParams ) then (

      Variables : SolverParams[1],

      /* Make sure that Variables is a list. Abort if it is not. */

      if not listp( Variables ) then
        ErrorHandler( "VarNotLst", Variables, 'Fatal ),

      /* delete multiple occurrences of identical symbols */

      Variables : Setify( Variables ),

      PrintMsg( 'DETAIL, SolverMsg["VarsAre"], Variables ),

      if length( SolverParams ) > 1 then (

        Parameters : SolverParams[2],

        /* Make sure that Parameters is a list. Abort if it is not. */

        if not listp( Parameters ) then
          ErrorHandler( "ParNotLst", Parameters, 'Fatal ),

        /* delete multiple occurrences of identical symbols */

        Parameters : Setify( Parameters ),

        PrintMsg( 'DETAIL, SolverMsg["ParsAre"], Parameters )
      )

    ),


    /* Check if Variables and Parameters are disjoint sets of symbols */

    if not DisjointP( Variables, Parameters ) then
      ErrorHandler(
        "VarParConfl", Intersection( Variables, Parameters ), 'Fatal
      ),


    /* Make a list of all variables to be solved for. This list may contain */
    /* more symbols than Variables when either no SolverParams have been    */
    /* given or when the user has specified only a subset of all existing   */
    /* symbols.                                                             */

    AllVars : SetDifference( listofvars( Equations ), Parameters ),

    /* solve for all available variables if Variables is empty. */

    if Empty( Variables ) then
      Variables : AllVars,


    /* Check all those equations which contain only equations of the form  */
    /* NUMBER = NUMBER or which contain only parameters and none of the    */
    /* variables of interest for consistency. Remove consistent equations  */
    /* and store the assumptions made in the list Solver_Assumptions.      */

    Equations : ParamConsistency( Equations, Parameters ),

    /* Abort if no equations are left after the above step. */

    if Empty( Equations ) or Empty( Variables ) then (
      PrintMsg( 'SHORT, SolverMsg["NoEqOrVar"] ),
      return( [ [], SolverParams, Variables, Parameters, [], [], [] ] )
    ),

    /* If there are compound expressions to be solved for then the solver     */
    /* tries to solve for the variables contained in them first.              */
    /* Subsequently the expressions are rebuilt from the solutions of         */
    /* these variables and the parameters.                                    */

    for i thru length( Variables ) do

      /* Search the variable list for non-atomic expressions. */

      if not atom( Var : Variables[i] ) then (

        /* append expression to the expression list */
        Expressions : endcons( Var, Expressions ),

        /* Insert the variables in the expression into the list of variables  */
        /* but keep the parameters out.                                       */
        Variables[i] : SetDifference( listofvars( Var ), Parameters ),

        PrintMsg( 'SHORT, SolverMsg["TrySolve4"], Variables[i] ),
        PrintMsg( 'SHORT, SolverMsg["Solve4Exp"], Var )
      ),

    /* Make a list of all the variables the user actually wants to know */
    UserVars : sublist( Variables, 'atom ),

    /* Flatten the list of variables and make it a set. */
    Variables : Setify( Flatten( Variables ) ),


    /* If Solver_Subst_Powers is true then substitute powers by new symbols */

    if Solver_Subst_Powers then (

      PrintMsg( 'SHORT, SolverMsg["SubstPwrs"] ),

      for Var in AllVars do (

        /* get all powers of Var in Equations. Convert negative powers to */
        /* positive ones.                                                 */
        Power : Setify( abs( ListOfPowers( Equations, Var ) ) ),

        /* If there is more than one power of Var then substitute each  */
        /* var^m only if all m are integer multiples of the lowest      */
        /* power > 0. The check is done by examining the modulus of all */
        /* powers with respect to the lowest power.                     */

        if
          not member( 'false, map( 'integerp, Power ) )
          and
          Power[1] # 1
        then

          if ( length( Power ) = 1 ) or

            block(
              [ Modulus : Power[1] ],
              not member(
                'false,
                map( 'ZeroP, totaldisrep( rat( rest( Power ) ) ) )
              )
            )

          then (
            /* Make a new symbol for var^power */
            SubstSym : concat( Var, "^", Power[1] ),

            /* Store a reference to the original term in an assoc list */
            PowerSubst : endcons( SubstSym = Var^Power[1], PowerSubst ),

            /* Substitute the new symbol for the original term */
            Equations  : ratsubst( SubstSym, Var^Power[1], Equations ),
            Variables  : subst( SubstSym, Var, Variables ),

            /* Notify user */
            PrintMsg( 'DETAIL, SolverMsg["subst"], Var^Power )
          )

      ) /* END for Var in AllVars */

    ), /* END if Solver_Subst_Powers */

    return(
      [
        Equations, SolverParams, Variables, Parameters, Expressions,
        PowerSubst, UserVars
      ]
    )
  )
)$


/******************************************************************************/
/* ParamConsistency checks equations of the form NUMBER = NUMBER and equa-    */
/* tions which contain only parameters for consistency. If the system cannot  */
/* determine whether a parametric expression is zero then the user is optio-  */
/* nally asked to supply the required information. The assumptions made are   */
/* stored in the list Solver_Assumptions.                                     */
/******************************************************************************/

ParamConsistency( Eqs, Pars, [ Action ] ) := (

  mode_declare(
    [ Eqs, Pars ], list,
    Action, any
  ),

  block(

    [ Eq, lhsminusrhs, i, consistent ],

    mode_declare(
      [ Eq, lhsminusrhs ], any,
      i, fixnum,
      consistent, boolean
    ),

    PrintMsg( 'SHORT, SolverMsg["ConsChk"] ),

    if Empty( Action ) then
      Action : 'BREAK
    else
      Action : Action[1],

    consistent : true,

    for i thru length( Eqs ) do (

      Eq : Eqs[i],

      /* Does the equation contain only numbers or parameters? */
      if Empty(
        SetDifference( listofvars( Eq ), Pars )
      )
      then (

        /* If so, check for consistency */
        lhsminusrhs :  expand( lhs( Eq ) - rhs( Eq ) ),

        /* Test if difference of both rhs's is zero */
        if lhsminusrhs # 0 then

          if SolverAssumeZero( lhsminusrhs ) then
            PrintMsg( 'SHORT, SolverMsg["Assum"], lhsminusrhs = 0 )
          else
            if Action = 'BREAK then (
              /* Abort if difference is non-zero */
              PrintMsg( 'SHORT, SolverMsg["Incons"], lhs( Eq ) = rhs( Eq ) ),
              TerminateSolver()
            )
            else
              return( consistent : false ),

        /* Kill the now redundant equation */
        if Action = 'BREAK then
          Eqs[i] : []

      ) /* END if Empty */

    ), /* END for i */

    if consistent then
      PrintMsg( 'SHORT, SolverMsg["NoneFnd"] ),

    if Action = 'BREAK then
      return( delete( [], Eqs ) )
    else
      return( consistent )
  )
)$


/******************************************************************************/
/* SolverAssumeZero checks whether Expression is (assumed to be) equal to     */
/* zero. If it isn't, the function returns false or asks the user for his     */
/* decision. New assumptions are appended to the list Solver_Assumptions.     */
/******************************************************************************/

SolverAssumeZero( Expression ) := (

  mode_declare(
    Expression, any
  ),

  block(

    [ AssumptionExists, i ],

    mode_declare(
      i, fixnum,
      AssumptionExists, boolean
    ),

    /* Return false immediately if Expression is a number # 0 or if  */
    /* Solver_Incons_Params is set to 'BREAK.                          */

    if
      Empty( listofvars( Expression ) )
      or ( Solver_Incons_Params = 'BREAK )
    then
      return( false ),

    /* Do a simple check to find out whether assumption already exists:    */
    /* Zero is substituted for Expression in the stored assumption. If the */
    /* result is zero then the assumption already exists.                  */

    AssumptionExists : false,
    for i thru length( Solver_Assumptions ) while not AssumptionExists do
      if
        fullratsimp(
          ratsubst( 0, Expression, lhs( Solver_Assumptions[i] ) )
        ) = 0
      then
        AssumptionExists : true,

    /* If the expression is not yet assumed to be equal to zero, ask the user */
    /* to decide whether it is.                                               */

    if not AssumptionExists then

      if
        ( Solver_Incons_Params = 'ASK )
        and
        ( AskZeroNonzero( Expression ) # 'zero )
      then
        return( false )
      else
        /* If difference is equal to zero or assumed to be so then store    */
        /* the constraint in the global list Solver_Assumptions. So the user*/
        /* has access to all assumptions made during the solution process   */
        /* and can check any numerical solutions for consistency with the   */
        /* assumptions.                                                     */
        Solver_Assumptions : endcons(  Expression = 0, Solver_Assumptions )

    else
      PrintMsg( 'DETAIL, SolverMsg["AssmFnd"], Expression = 0 ),

    return( true )
  )
)$

AskZeroNonzero( Expression ) :=
  if equal( x, 0 ) = true then 'zero
  else block(
    [s : 'pnz],
    while s#'zero and s#'nonzero do (
      s : read( "Is", Expression, "zero or nonzero?" )
    ),
    s
  )$

/******************************************************************************/
/* ListOfPowers returns the list of powers # 0 of a variable in a set of      */
/* equations.                                                                 */
/******************************************************************************/

ListOfPowers( Eqs, Var ) := (

  mode_declare(
    Eqs, list,
    Var, any
  ),

  delete(
    0,
    apply(
      'Union,
      map(
        lambda(
          [ Eq ],
          Powers( expand( lhs( Eq ) - rhs( Eq ) ), Var )
        ),
        Eqs
      )
    )
  )
)$


/******************************************************************************/
/* ImmediateAssignments directly applies all immediate assignments of the     */
/* form  var = rhs...  before the actual Solver is called.                    */
/******************************************************************************/

ImmediateAssignments( Solutions, RemainingEqs, Variables, Parameters ) := (

  mode_declare(
    [ Solutions, RemainingEqs, Variables, Parameters ], list
  ),

  block(

    [ i, Vars, AssignmentMade, Left, Right ],

    mode_declare(
      i, fixnum,
      Vars, list,
      AssignmentMade, boolean,
      [ Left, Right ], any
    ),

    PrintMsg( 'SHORT, SolverMsg["SrchImmed"] ),

    AssignmentMade : false,

    for i thru length( RemainingEqs ) do block(
      [],

      Left  : lhs( RemainingEqs[i] ),
      Right : rhs( RemainingEqs[i] ),

      /* Scan the equations for simple assignments of the form  X = Expr or   */
      /* Expr = X  and apply this assignment only if X is not a parameter and */
      /* if Expr contains only numbers or parameters.                         */
      /* Remark: Equations containing only parameters have already been remo- */
      /* ved from the equation list in SetupSolver.                           */

      if symbolp( Left ) then (

        Vars : listofvars( Right ),

        /* Check if lhs is an isolated variable and make sure that there are */
        /* only parameters on the right-hand side.                           */

        if freeof( Left, Right )
          and Empty( SetDifference( Vars, Parameters ) )
        then (

          if assoc( Left, Solutions ) = false then (
            PrintMsg(
              'DETAIL,
              SolverMsg["Assign"], totaldisrep( Left = Right )
            ),

            if member( Left, map( lhs, Solutions ) ) then (
              if assoc( Left, Solutions ) # Right then (
                PrintMsg( 'SHORT, SolverMsg["Incons"], Left = Right ),
                TerminateSolver()
              )
            )
            else
              Solutions : endcons( Left = Right, Solutions ),
            RemainingEqs[i] : [],

            AssignmentMade : true
          ),

          /* This return prevents Macsyma from entering the next if statement */
          /* which must only be executed when the current outer if statement  */
          /* has not been entered.                                            */
          return( 'DONE )
        )
      ),

      if symbolp( Right ) then (

        Vars : listofvars( Left ),

        if freeof( Right, Left )
          and Empty( SetDifference( Vars, Parameters ) )
        then (
        
          if assoc( Right, Solutions ) = false then (
            PrintMsg(
              'DETAIL,
              SolverMsg["Assign"], totaldisrep( Right = Left )
            ),
          
            if member( Right, map( lhs, Solutions ) ) then (
              if assoc( Right, Solutions ) # Left then (
                PrintMsg( 'SHORT, SolverMsg["Incons"], Left = Right ),
                TerminateSolver()
              )
            )
            else
              Solutions : endcons( Right = Left, Solutions ),
            RemainingEqs[i] : [],

            AssignmentMade : true
          )
          
        )
      )
    ), /* END for i thru length */

    if AssignmentMade then (

      /* delete the used equations from the list */
      RemainingEqs : delete( [], RemainingEqs ),

      if not Empty( RemainingEqs ) then (

        /* Remove all the variables from the working list which have been */
        /* determined by an immediate assignment.                         */
        Variables : SetDifference( Variables, map( 'lhs, Solutions ) ),

        /* Evaluate the remaining equations with the constraints */
        RemainingEqs : ev( RemainingEqs, Solutions ),

        /* Do a parameter consistency check */
        RemainingEqs : ParamConsistency( RemainingEqs, Parameters )
      )

      else
        PrintMsg( 'SHORT, SolverMsg["NoEqTerm"] )

    )
    else
      PrintMsg( 'SHORT, SolverMsg["NoImmed"] ),

    /* END if AssignmentMade */

    return( [ AssignmentMade, Solutions, RemainingEqs, Variables ] )
  )
)$


/******************************************************************************/
/* LinearSolver extracts blocks of linear equations from an arbitrary system  */
/* of equations by a heuristic searching strategy and solves the linear block */
/* if there is one.                                                           */
/******************************************************************************/

LinearSolver( Solutions, Equations, Variables, Parameters ) := (

  mode_declare(
    [ Solutions, Equations, Variables, Parameters ], list
  ),

  block(

    [
      CoeffMatrix, ValuationMatrix, ActiveVars,
      LinEqNos, LinVarNos, NewVars, LinSolVars,
      LinearEqs, LinearVars, LinearSolutions,
      i, j, me, mv, NumVars, NumEqs,
      MaxValVar, MaxValEq,
      EqValuation, VarValuation,
      rhsExpressions,

      linsolvewarn             : false,
      linsolve_params          : false,
      Solve_Inconsistent_Error : false,

      EqWasLast : false
    ],

    mode_declare(
      [ CoeffMatrix, ValuationMatrix ], any,
      [
        LinEqNos, LinVarNos, ActiveVars, LinearEqs, LinearVars, LinSolVars,
        EqValuation, VarValuation, LinearSolutions, NewVars, rhsExpressions
      ], list,
      [ i, j, me, mv, NumVars, NumEqs, MaxValVar, MaxValEq ], fixnum
    ),

    if Empty( Equations ) then (

      if Empty( Variables ) then
        PrintMsg( 'SHORT, SolverMsg["AllSolved"] )
      else
        PrintMsg( 'SHORT, SolverMsg["NoEqLeft"], Variables ),

      return( [ false, Solutions, Equations, Variables ] )
    )
    else if Empty( Variables ) then (
      PrintMsg( 'SHORT, SolverMsg["EqLeft"] ),
      return( [ false, Solutions, Equations, Variables ] )
    ),

    PrintMsg( 'SHORT, SolverMsg["SrchLinEq"] ),

    /* Make a list of all remaining variables */
    NewVars : listofvars( Equations ),

    if Solver_Find_All_Linear_Vars then (

      /* The following rather weird commands put the variables into the order */
      /* [ variables to be currently solved for, other variables ]. If the    */
      /* current variables are linear variables then it is more likely that   */
      /* they will have explicit solutions if they are located in the left    */
      /* half of the system of equations. Otherwise the will more likely be   */
      /* used as parameters of the null space (if there is one).              */

      ActiveVars : append(
        Intersection( Variables, NewVars ),
        SetDifference( NewVars, append( Variables, Parameters ) )
      ),
      LinearEqs : Equations,
      Equations : []
    )
    else (
      ActiveVars : Intersection( Variables, NewVars ),
      LinearEqs : [],
      for i thru length( Equations ) do
        if
          not Empty( Intersection( listofvars( Equations[i] ), ActiveVars ) )
        then (
          LinearEqs : endcons( Equations[i], LinearEqs ),
          Equations[i] : []
        ),
      Equations : delete( [], Equations )
    ),

    PrintMsg( 'SHORT, SolverMsg["wrt"], ActiveVars ),

    /* Set up the complete coefficient matrix w.r.t. all variables */
    CoeffMatrix : ComplCoeffMatrix( LinearEqs, ActiveVars ),

    /* The valuation matrix contains a 1 at each Position where a variable */
    /* appears in a nonlinear form, and 0's otherwise.                     */

    ValuationMatrix : matrixmap(
      lambda(
        [ x ],
        if x = 'false then
          1
        else
          0
      ),
      CoeffMatrix
    ),

    /* EqValuation contains the number of nonlinear variables for each eq. */
    EqValuation : map(
      lambda( [ Row ], apply( "+", Row ) ),
      ListMatrix( ValuationMatrix )
    ),

    /* VarValuation contains for all vars the number of equations in which */
    /* var[i] appears in a nonlinear form.                                 */
    VarValuation : map(
      lambda( [ Row ], apply( "+", Row ) ),
      ListMatrix( transpose( ValuationMatrix ) )
    ),

    LinEqNos  : makelist( i, i, 1, NumEqs : length( EqValuation ) ),
    LinVarNos : makelist( i, i, 1, NumVars : length( VarValuation ) ),


    /* Remove nonlinear equations and/or variables until only a linear    */
    /* block remains, i.e. for all i, j VarValuation[i] = 0 and           */
    /* EqValuation[j] = 0.                                                */

    while
      ( apply( "+", VarValuation ) # 0 )
      and
      ( apply( "+", EqValuation ) # 0 )
    do (

      /* Determine maximum equation valuation and number of corresponding */
      /* equation.                                                        */

      me : 0,
      MaxValEq : -1,
      for i thru NumEqs do
        if EqValuation[i] > MaxValEq then (
          me : i,
          MaxValEq : mode_identity( fixnum, EqValuation[i] )
        ),

      /* Determine maximum variable valuation and number of corresponding */
      /* variable.                                                        */

      mv : 0,
      MaxValVar : -1,
      for j thru NumVars do
        if VarValuation[j] > MaxValVar then (
          mv : j,
          MaxValVar : mode_identity( fixnum, VarValuation[j] )
        ),


      if apply( Solver_Del_Eq, [ MaxValEq, MaxValVar ] ) then (
        i : me,

        for j thru NumVars do (
          VarValuation[j] : VarValuation[j] - ValuationMatrix[i, j],
          ValuationMatrix[i, j] : 0
        ),

        /* Mark equation as deleted */
        EqValuation[i] : 0,
        LinEqNos[i]    : 0
      )
      else (
        j : mv,

        for i thru NumEqs do (
          EqValuation[i] : EqValuation[i] - ValuationMatrix[i, j],
          ValuationMatrix[i, j] : 0
        ),

        /* Mark variable as deleted */
        VarValuation[j] : 0,
        LinVarNos[j]    : 0
      )

    ), /* END while */

    /* Make list of linear equations */
    LinEqNos  : delete( 0, LinEqNos ),

    /* Make list of linear variables */
    LinVarNos : delete( 0, LinVarNos ),

    if Empty( LinEqNos ) or Empty( LinVarNos ) then (
      PrintMsg( 'SHORT, SolverMsg["NoLinEqs"] ),
      return( [ false, Solutions, append( LinearEqs, Equations ), Variables ] )
    ),

    /* Extract linear equations. append all nonlinear equations to Equations */
    /* again.                                                                */
    for i thru length( LinearEqs ) do
      if not member( i, LinEqNos ) then (
        Equations : endcons( LinearEqs[i], Equations ),
        LinearEqs[i] : []
      ),

    LinearEqs : delete( [], LinearEqs ),


    /* Extract linear variables. Since the extraction of linear equations may */
    /* also have removed linear variables (the coefficients are now 0) it is  */
    /* necessary to intersect the set of the linear variables with the set of */
    /* those variables which actually appear in the linear equations.         */
  
    LinearVars : Intersection(
      map( lambda( [ i ], ActiveVars[i] ), LinVarNos ),
      listofvars( LinearEqs )
    ),

    /* By analogy, the same applies to the linear equations. Thus, keep only  */
    /* those equations which still contain any of the linear variables.       */

    for i thru length( LinearEqs ) do
      if DisjointP( listofvars( LinearEqs[i] ), LinearVars ) then (
        Equations : endcons( LinearEqs[i], Equations ),
        LinearEqs[i] : []
      ),

    LinearEqs : delete( [], LinearEqs ),

    /* Return if no linear equations are left */

    if Empty( LinearVars ) or Empty( LinearEqs ) then (
      PrintMsg( 'SHORT, SolverMsg["NoLinEqs"] ),
      return( [ false, Solutions, append( LinearEqs, Equations ), Variables ] )
    ),

    PrintMsg(
      'SHORT,
      SolverMsg["Found"], length( LinearEqs ), SolverMsg["LinEqs"],
      length( LinearVars ), SolverMsg["LinVars"]
    ),
    PrintMsg( 'SHORT,  SolverMsg["VarsAre"],  LinearVars ),
    PrintMsg( 'DETAIL, SolverMsg["EqsAre"],   LinearEqs ),
    PrintMsg( 'SHORT,  SolverMsg["SolvLinEq"] ),

    rhsExpressions : [],
    Solve_Inconsistent_Eqn_Nos : [ 0 ],

    while not Empty( Solve_Inconsistent_Eqn_Nos ) do (

      /* solve the linear equations */
      LinearSolutions : LinsolveM( LinearEqs , LinearVars ),

      /* Check for "inconsistent" equations. */
      if not Empty( Solve_Inconsistent_Eqn_Nos ) then (

        PrintMsg( 'DEBUG, SolverMsg["Incons"], Solve_Inconsistent_Eqn_Nos ),

        /* Remove "inconsistent" equations */
        for i in Solve_Inconsistent_Eqn_Nos  do
            LinearEqs[i] : [],

        LinearEqs : delete( [], LinearEqs ),

        /* append rhs = 0 to Solver_Assumptions if rhs contains only */
        /* parameters.                                               */
        rhsExpressions : ParamConsistency(
          Solve_Inconsistent_Terms, Parameters
        )

      )

    ),

    /* append rhs's which have led to "inconsistencies" but still   */
    /* contain variables to the list of equations.                  */
    Equations : append( Equations, rhsExpressions ),

    PrintMsg( 'DETAIL, SolverMsg["Solutions"], LinearSolutions ),

    /* Insert the solutions from linsolve into the remaining equations */
    Equations : ParamConsistency(
        fullratsimp( ev( Equations, LinearSolutions ) ),
        Parameters
    ),

    /* append the linear solutions to the list of solutions */
    Solutions : append( Solutions, LinearSolutions ),

    /* append all variables to the working list which appear on the rhs's of  */
    /* the linear solutions. delete all variables which have been solved for. */

    /* Linear variables for which a solution has been obtained. */
    LinSolVars : map( 'lhs, LinearSolutions ),

    /* Linear variables which are free parameters of the null space. */
    LinearVars : SetDifference( LinearVars, LinSolVars ),

    /* append all those variables which are parameters of the null space of */
    /* the linear equations and which do not appear in the remaining        */
    /* equations to the list of parameters.                                 */

    for Var in LinearVars do
      if freeof( Var, Equations ) then (
        PrintMsg( 'SHORT, SolverMsg["FreeVar2Par"], Var ),
        Parameters : endcons( Var, Parameters ),
        Variables  : delete( Var, Variables )
      ),

    NewVars : SetDifference(
      listofvars( map( 'rhs, LinearSolutions ) ),
      Parameters
    ),

    Variables : Union(
      SetDifference( Variables, LinSolVars ),
      NewVars
    ),

    return( [ true, Solutions, Equations, Variables ] )
  )
)$


/******************************************************************************/
/* The following strategies decide whether the linear solver should delete a  */
/* nonlinear equation or a nonlinear variable from the system while searching */
/* for linear subblocks of equations.                                         */
/******************************************************************************/

define_variable( EqWasLast, false, boolean )$

MakeSquareLinearBlocks( ValEq, ValVar ) := (

  mode_declare(
    [ ValEq, ValVar ], fixnum
  ),

  if ValEq = ValVar then
    EqWasLast : not EqWasLast
  else
    if ValEq > ValVar then
      EqWasLast : true
    else
      EqWasLast : false
)$


DelEqBeforeVar( ValEq, ValVar ) := (

  mode_declare(
    [ ValEq, ValVar ], fixnum
  ),

  if ValEq >= ValVar then
    true
  else
    false
)$


/******************************************************************************/
/* ComplCoeffMatrix returns a matrix whose row size is equal to the number of */
/* equations and whose column size is equal to the number of variables. The   */
/* entry at Position [i,j] is RatCoeff( equation[i], variable[j] ) if         */
/* equation[i] is linear w.r.t. variable[j] and 'false if equation[i] is      */
/* nonlinear w.r.t. variable[j].                                              */
/******************************************************************************/

ComplCoeffMatrix( Eqs, ActiveVars ) := (

  mode_declare(
    [ Eqs, ActiveVars ], list
  ),

  block(


    apply(

      'matrix,

      /* for each equation do */
      map(

        lambda(
          [ Eq ],

          /* for each variable do */
          map(
            lambda(
              [ Var ],
              block(
                [ rc ],
                rc  : LinCoeff( Eq, Var ),

                /* Return the RatCoeff only if it contains none of the active */
                /* variables or if the equation doesn't contain var at all.   */

                if
                  ( ( rc # 0 ) and DisjointP( listofvars( rc ), ActiveVars ) )
                  or
                  freeof( Var, Eq )
                then
                  rc
                else
                  false
              )
            ),

            ActiveVars
          ) /* END lambda( [ Var ] ) */

        ), /* END lambda( [ Eq ] ) */


        /* map target: Transform all equations into homogeneous form. */

        map(
          lambda(
            [ Eq ],
            fullratsimp( expand( lhs( Eq ) - rhs( Eq ) ) )
          ),
          Eqs
        )


      ) /* END map( lambda( [ Eq ] ) ) */

    ) /* END apply */
  )
)$


/******************************************************************************/
/* LinCoeff returns the linear coefficient of Var within Eq if Var appears    */
/* raised to the first power only.                                            */
/******************************************************************************/

LinCoeff( Eq, Var ) := (

  mode_declare(
    [ Eq, Var ], any
  ),

  block(

    [ BCoeff ],

    mode_declare(
      BCoeff, list
    ),

    if ListOfPowers( [ Eq ], Var ) = [ 1 ] then (
      BCoeff : bothcoeff( Eq, Var ),
      if freeof( Var, second( BCoeff ) ) then
        return( first( BCoeff ) )
    ),

    return( 0 )
  )
)$


/******************************************************************************/
/* ValuationSolver                                                            */
/******************************************************************************/

ValuationSolver( Solutions, Equations, Variables, Parameters ) := (

  mode_declare(
    [ Solutions, Equations, Variables, Parameters ], list
  ),

  block(

    [
      VarPaths, ValMatrix, Eq, Var, Trans, TempEq, TransEq,
      SolveOrder, SolveInfo, Transform, Solution, SolCheck,
      Status, Solved, Failed, UniqueSol, TryToSolve, CheckSol,
      i, k
    ],

    mode_declare(
      [ VarPaths, ValMatrix, Eq, Var, Trans, TempEq, TransEq ], any,
      [ SolveOrder, SolveInfo, Transform, Solution, SolCheck ], list,
      [ Status, Solved, Failed, UniqueSol, TryToSolve, CheckSol ], boolean,
      [ i, k ], fixnum
    ),


    UniqueSol : true,

    LOOP,

    PrintMsg( 'SHORT, SolverMsg["Chk4RemEq"] ),

    if Empty( Equations ) then (

      if Empty( Variables ) then
        PrintMsg( 'SHORT, SolverMsg["AllSolved"] )
      else
        PrintMsg( 'SHORT, SolverMsg["NoEqLeft"], Variables ),

      Status : false

    )
    else if Empty( Variables ) then (
      PrintMsg( 'SHORT, SolverMsg["EqLeft"] ),
      Status : false
    )
    else (

      PrintMsg(
        'SHORT,
        length( Equations ), SolverMsg["Eqs"],
        length( Variables ), SolverMsg["Vars"]
      ),
      PrintMsg( 'DETAIL, SolverMsg["VarsAre"], Variables ),
      PrintMsg( 'DEBUG, SolverMsg["EqsAre"], Equations ),

      /* Dump solutions and remaining equations to file if requested. */
      if Solver_Dump_To_File then
        DumpToFile( Solutions, Equations, Variables ),

      if ( length( Variables ) = 1 ) and ( length( Equations ) = 1 ) then
        SolveOrder : [ [ 1, 1, "(irrelevant)" ] ]

      else (
        PrintMsg( 'SHORT, SolverMsg["ValStrat"] ),

        /* Set up the valuation matrices. */
        VarPaths  : OccurrenceMatrix( Equations, Variables ),
        ValMatrix : ValuationMatrix( Equations, Variables ),

        /* Determine an order by which the equations should be solved. */
        SolveOrder : apply( Solver_Valuation_Strategy, [ VarPaths, ValMatrix ] )
      ),

      Solved : false,

      unless Solved or Empty( SolveOrder ) do (

        SolveInfo : Pop( SolveOrder ),
        if listp(SolveInfo[1]) then SolveInfo : first(SolveInfo),
        k   : mode_identity( fixnum, first( SolveInfo ) ),
        Eq  : part( Equations, k ),
        Var : Variables[ second( SolveInfo ) ],

        PrintMsg( 
          'SHORT, 
          SolverMsg["TrySolveEq"], k, SolverMsg["ForVar"], Var 
        ),
        PrintMsg( 'SHORT, SolverMsg["Valuation"], third( SolveInfo ) ),
        PrintMsg( 'DETAIL, SolverMsg["EqIs"], Eq = 0 ),

        TryToSolve : true,
        CheckSol   : true,
        Transform  : copylist( Solver_Transforms ),

        /* Do the solver break test to check whether it is worth */
        /* attempting to solve the equation at all.              */
        Failed : apply( Solver_Break_Test, [Eq, Var, third( SolveInfo )] ),
        
        unless Solved or Failed do (

          /* Try to solve the selected equation */
          if TryToSolve then
            Solution : solve( Eq, Var ),

          /* Check if the equation was solved correctly */
          if CheckSol then (
            PrintMsg( 'SHORT, SolverMsg["CheckSol"] ),
            SolCheck : SolutionOK( Solution, Var ),

            PrintMsg( 'DETAIL, SolverMsg["Solutions"], Solution )
          )
          else
            SolCheck : [ false ],

          /* All solutions OK? */
          if member( true, SolCheck ) then (
            PrintMsg( 'SHORT, SolverMsg["SolOK"] ),
            Solved : true
          )

          /* If not, apply transformations */
          else (
            if CheckSol then
              PrintMsg( 'SHORT, SolverMsg["SolNotOK"] ),

            /* Give up if no transformations are left */
            if Empty( Transform ) then (
              PrintMsg( 'SHORT, SolverMsg["GiveUp"] ),
              Failed : true
            )

            else (
              /* Retrieve one transformation function */
              Trans : Pop( Transform ),
              PrintMsg( 'SHORT,  SolverMsg["AppTrans"], Trans ),

              /* and apply it to the equation, the variable, and the solution */
              TransEq : apply( Trans, [ Eq, Var, Solution ] ),

              /* The transformation should return an equation as its function */
              /* value. However, if no reasonable transformation of the       */
              /* equation was possible then the SOLVE function should not be  */
              /* tried again. Hence, to signal a failure, the transformation  */
              /* must return an empty list, which will instruct the Solver to */
              /* try the next transformation instead. In addition, the        */
              /* transformation may itself take care of solving the equation. */
              /* It must then return a list of solutions:                     */
              /*   [ var = solution_1, var = solution_2, ... ]                */

              /* Did the transformation fail? */
              if TransEq = [] then (
                PrintMsg( 'SHORT, SolverMsg["TransFail"] ),

                /* Instruct the Solver to try the next transformation */
                TryToSolve : false,
                CheckSol   : false
              )

              /* Did it solve the equation by itself? */
              else if listp( TransEq ) then (
                PrintMsg( 'SHORT, SolverMsg["TransSolv"] ),
                Solution : TransEq,

                /* Instruct the Solver not to call SOLVE again */
                TryToSolve : false,
                CheckSol   : true
              )

              /* Transformation thinks it has succeeded, so try again */
              else (
                Eq : TransEq,
                PrintMsg( 'DETAIL, SolverMsg["ResTrans"], Eq = 0 ),
                PrintMsg( 'SHORT,  SolverMsg["RetryTrans"] ),

                TryToSolve : true,
                CheckSol   : true
              )

            ) /* END if Empty( Transform ) else */

          ) /* if member( true, SolCheck ) else */

        ) /* END unless Solved or Failed */

      ), /* END unless Solved or Empty( SolveOrder ) */

      if Solved then (
        if length( Solution ) > 1 then
          PrintMsg( 'SHORT, SolverMsg["NotUnique"] ),

        if member( false, SolCheck ) then
          PrintMsg( 'SHORT, SolverMsg["SolsLost"] ),

        /* Remove solved equation from list of equations. Store it in TempEq  */
        /* so it can be appended to Equations again if the consistency check  */
        /* fails.                                                             */
        TempEq : part( Equations, k ),
        Equations : delete( [], Set_Element( Equations, k, [] ) ),

        /* Check solutions for consistency with remaining equations. */
        for i thru length( Solution ) do (

          if part( SolCheck, i ) then (

            PrintMsg(
              'DETAIL, SolverMsg["Solution"], i, SolverMsg["ForVar"], Var
            ),

            if not ParamConsistency(
              fullratsimp( ev( Equations, part( Solution, i ) ) ),
              Parameters,
              'CONTINUE
            ) then (
              PrintMsg( 'SHORT, SolverMsg["Contradict"], part( Solution, i ) ),
              Set_Element( Solution, i, 'INCONSISTENT_PATH )
            )

          )
          else (
            PrintMsg( 'DETAIL, SolverMsg["Dropped"], part( Solution, i ) ),
            Set_Element( Solution, i, [] )
          )

        ),

        /* delete all implicit or empty solutions */
        Solution : delete( [], Solution ),

        if Empty( Solution ) then (
          PrintMsg( 'SHORT, SolverMsg["NoValidSol"], Var ),

          Equations : endcons( TempEq, Equations ),
          Solved : false
        ) 

        /* Check if there are any consistent solutions */
        else if not member( 
          true, 
          map( 
            lambda( 
              [x], 
              if x = 'INCONSISTENT_PATH then
                false
              else
                true
            ), 
            Solution
          ) 
        ) 
        then (
          PrintMsg( 'SHORT, SolverMsg["NoConsSol"], Var ),

          Solutions : endcons( 'INCONSISTENT_PATH, Solutions ),
          Solved : false
        )
 
        /* append consistent solutions to the solution list */
        else (
          PrintMsg( 'DETAIL, SolverMsg["ConsSol"], Var, ":", Solution ),

          Variables : delete( Var, Variables ),

          /* If the solution is unique or if there's only one consistent */
          /* solution then ...                                           */
          if length( Solution ) = 1 then (

            /* ... store it, insert it into the remaining equations, and */
            /* add its rhs variables to the list of unknowns.            */
            Solutions : append( Solutions, Solution ),
            Equations : copylist(
              fullratsimp( ev( Equations, Solution ) )
            ),
            Variables : Union(
              Variables,
              SetDifference(
                listofvars( rhs( first( Solution ) ) ),
                Parameters
              )
            )
          )
          
          else /* length( Solution ) > 1, call ValuationSolver recursively */

            block(
              [ MultipleSolutions, RSolutions, RVars, REqs, Sol, Stat ],

              mode_declare(
                [ MultipleSolutions, RSolutions, RVars, REqs ], list,
                Sol, any,
                Stat, boolean
              ),

              MultipleSolutions : [],

              for Sol in Solution do (

                map(
                  lambda([x,y], x::y),
                  [ 'Stat, 'RSolutions, 'REqs, 'RVars ],
                  ValuationSolver(
                    [ Sol ],
                    copylist( fullratsimp( ev( Equations, Sol ) ) ),
                    Union(
                      Variables,
                      SetDifference( listofvars( rhs( Sol ) ), Parameters )
                    ),
                    Parameters
                  )
                ),

                MultipleSolutions : endcons( RSolutions, MultipleSolutions )

              ), /* END for Sol */

              Solutions : endcons( MultipleSolutions, Solutions ),

              UniqueSol : false
            ) /* END block */

        ) /* END if Empty( Solution ) */

      )

      else (

        /* append remaining equations to the solutions if no further */
        /* solutions could be determined.                            */

        for e in Equations do (
          if is( equal( e, 0 ) ) # 'unknown then (
            PrintMsg( SHORT, "Inconsistent equation", e = 0),
            TerminateSolver()
          )
        ),

        Solutions : endcons( [ Equations ], Solutions )
      ), /* END if Solved */

      Status : Solved
    ),

    if Status and UniqueSol then
      go( LOOP )
    else
      return( [ Status, Solutions, Equations, Variables ] )
  )
)$


/******************************************************************************/
/* SolutionOK checks whether the result of a call to the solve function is    */
/* indeed a solution of the form  var = expression_free_of_var.               */
/******************************************************************************/

SolutionOK( Solution, Var ) := (

  mode_declare(
    [ Solution, Var ], any
  ),

  if listp( Solution ) then

    /* List of solutions must not be empty. */
    if Empty( Solution ) then
      [ false ]

    else

      /* Check if the lhs of each solution is equal to var and make sure */
      /* that var does not appear on the rhs's.                          */
      map(
        lambda(
          [ Sol ],
          ( lhs( Sol ) = Var ) and freeof( Var, rhs( Sol ) )
        ),
        Solution
      )

  else
    [ false ]

)$


/******************************************************************************/
/* ValuationMatrix generates a matrix of valuations with respect to each      */
/* equation and each variable.                                                */
/******************************************************************************/

ValuationMatrix( Equations, Variables ) := (

  mode_declare(
    [ Equations, Variables ], list
  ),

  genmatrix(
    lambda( [ i, j ], Valuation( Equations[i], Variables[j] ) ),
    length( Equations ), length( Variables )
  )
)$


/******************************************************************************/
/* Operator valuation factors for expression valuation.                       */
/******************************************************************************/

SetProp( 'sin,   'Valuation, 10 )$
SetProp( 'cos,   'Valuation, 10 )$
SetProp( 'tan,   'Valuation, 10 )$
SetProp( 'asin,  'Valuation, 12 )$
SetProp( 'acos,  'Valuation, 12 )$
SetProp( 'atan,  'Valuation, 12 )$
SetProp( 'sinh,  'Valuation, 12 )$
SetProp( 'cosh,  'Valuation, 12 )$
SetProp( 'tanh,  'Valuation, 12 )$
SetProp( 'asinh, 'Valuation, 12 )$
SetProp( 'acosh, 'Valuation, 12 )$
SetProp( 'atanh, 'Valuation, 12 )$
SetProp( "+",    'Valuation,  1 )$
SetProp( "-",    'Valuation,  1 )$
SetProp( "*",    'Valuation,  4 )$
SetProp( "/",    'Valuation,  4 )$
SetProp( "^",    'Valuation, 10 )$
SetProp( 'sqrt,  'Valuation, 10 )$
SetProp( 'exp,   'Valuation, 10 )$
SetProp( 'log,   'Valuation, 10 )$


/******************************************************************************/
/* With SetValuation, the operator valuation factors can be redefined.        */
/******************************************************************************/

SetValuation( Operator, Valuation ) :=
  SetProp( Operator, 'Valuation, Valuation )$
  

/******************************************************************************/
/* Valuation measures the complexity of an expression with respect to Var by  */
/* weighting the operator tree representation of Expr.                        */
/******************************************************************************/

Valuation( Expr, Var ) := (

  mode_declare(
    [ Expr, Var ], any
  ),


  block(

    [ OpFactor ],

    mode_declare(
      OpFactor, fixnum
    ),

    /* Return zero if Expr does not contain Var. */
    if freeof( Var, Expr ) then
      return( 0 )

    else
      /* Return 1 if Expr is an atom, i.e. Expr = Var. */
      if atom( Expr ) then
        return( 1 )

      /* If Expr is an algebraic expression then retrieve the valuation    */
      /* factor associated with the operator of Expr and recursively apply */
      /* the valuation function to each subexpression of Expr.             */
      else (

        if (
            OpFactor : mode_identity( fixnum, get( op( Expr ), 'Valuation ) )
           ) = false
        then
          OpFactor : Solver_Default_Valuation,

        return(
          OpFactor * apply(
            "+",
            map(
              lambda( [ SubExpr ], Valuation( SubExpr, Var ) ),
              substpart( "[", Expr, 0 )
            )
          )
        )

      ) /* END if atom */

  )
)$


/******************************************************************************/
/* OccurrenceMatrix sets up a matrix in which the number of occurrences of     */
/* each variable in each equation is counted.                                 */
/******************************************************************************/

OccurrenceMatrix( Equations, Variables ) := (

  mode_declare(
    [ Equations, Variables ], list
  ),

  genmatrix(
    lambda( [ i, j ], Occurences( Equations[i], Variables[j] ) ),
    length( Equations ), length( Variables )
  )
)$


/******************************************************************************/
/* Occurences counts the number of occurences of Var in Expr, i.e. the number */
/* of paths to distinct occurrences of the atom Var in the internal tree      */
/* representation of Expr.                                                    */
/******************************************************************************/

Occurences( Expr, Var ) := (

  mode_declare(
    [ Expr, Var ], any
  ),

  if atom( Expr ) then

    if Expr = Var then
      1
    else
      0

  else
    apply(
      "+",
      map(
        lambda( [ SubExpr], Occurences( SubExpr, Var ) ),
        substpart( "[", Expr, 0)
      )
    )
)$


/******************************************************************************/
/* MinVarPathsFirst tries to find variables which can be easily isolated.     */
/* These are variables which appear only once in an entire expression tree    */
/* (= 1 in OccurrenceMatrix).                                                 */
/******************************************************************************/

MinVarPathsFirst( OccMat, ValMat ) := (

  mode_declare(
    [ OccMat, ValMat ], any
  ),

  block(

    [
      SolveOrder, SolveOrder1, SumVarPaths,
      i, j, v, ne, nv
    ],

    mode_declare(
      [ SolverOrder, SolveOrder1, SumVarPaths ], list,
      [ i, j, v, ne, nv, Function( RowSize, ColSize, Position ) ], fixnum
    ),

    SolveOrder : [],

    ne : RowSize( OccMat ),
    nv : ColSize( OccMat ),

    SumVarPaths : map(
      lambda( [ Row ], apply( "+", Row ) ),
      ListMatrix( OccMat )
    ),

    /* Search for equations which contain only one variable in one path. */
    for i thru length( SumVarPaths ) do
      if SumVarPaths[i] = 1 then (
        SolveOrder : endcons(
          [ i, j : Position( 1, OccMat[i] ), ValMat[i, j] ],
          SolveOrder
        ),
        /* Mark eq/var Position as used. */
        OccMat[i, j] : 0,
        ValMat[i, j] : 0
      ),

    /* Sort SolveOrder by least valuation. */
    if not Empty( SolveOrder ) then
      SolveOrder : SortSolveOrder( SolveOrder ),

    /* Find all variables with only one path in the expression tree. */
    SolveOrder1 : [],
    for i thru ne do
      for j thru nv do
        if OccMat[i, j] = 1 then (
          SolveOrder1 : endcons( [ i, j, ValMat[i, j] ], SolveOrder1 ),
          OccMat[i, j] : 0,
          ValMat[i, j] : 0
        ),

    /* Sort variables by least valuation. */
    if not Empty( SolveOrder1 ) then
      SolveOrder : append(
        SolveOrder,
        SortSolveOrder( SolveOrder1 )
      ),

    /* append additional candidates if necessary. */
    if length( SolveOrder ) < Solver_Max_Len_Val_Order then (

      SolveOrder1 : [],
      for i thru ne do
        for j thru nv do
          if ( v : mode_identity( fixnum, ValMat[i, j] ) ) # 0 then
            SolveOrder1 : endcons( [ i, j, v ], SolveOrder1 ),

      SolveOrder : append(
        SolveOrder,
        SortSolveOrder( SolveOrder1 )
      ),

      /* Return only as many candidates as given by Solver_Max_Len_Val_Order */
      if ( i : length( SolveOrder ) ) > Solver_Max_Len_Val_Order then
        SolveOrder : rest(
          SolveOrder, Solver_Max_Len_Val_Order - i
        )
    ),

    return( SolveOrder )
  )
)$


/******************************************************************************/
/* PostProcess does all the postprocessing needed to display the results.     */
/* This includes expansion of the solution list hierarchies, backsubsitution, */
/* and extraction of the variables which the user explicitly asked for.       */
/******************************************************************************/

PostProcess( Solutions, UserVars, Expressions, PowerSubst ) := (

  mode_declare(
    [ Solutions, UserVars, Expressions, PowerSubst ], list
  ),

  block(
    [
      SolSet, UsrSolSet, UserSolutions, UnsolvedEqs, InternalSols,
      Var, TempVar, EvalVar, 
      i
    ],

    mode_declare(
      [ SolSet, UsrSolSet, UserSolutions, UnsolvedEqs, InternalSols ], list,
      [ Var, TempVar, EvalVar ], any,
      i, fixnum
    ),


    PrintMsg( 'SHORT, SolverMsg["PostPr"] ),

    /* first of all, Flatten the solution list hierarchy and drop all */
    /* inconsistent solution paths.                                   */
    Solutions : sublist(
      ExpandSolutionHierarchy( Solutions ),
      lambda( [Set], last( Set ) # 'INCONSISTENT_PATH )
    ),

    /* Return an empty list if no consistent solution paths are left. */
    if Empty( Solutions ) then
      return( [] ),

    /* Do the backsubstitutions. */
    if not Empty( Solutions ) then (

      UserSolutions : [],
      i : 0,

      for SolSet in Solutions do (

        i : i + 1,
        PrintMsg( 'SHORT, SolverMsg["SolSet"], i ),

        UsrSolSet   : [],

        /* Extract the unsolved equations */
        UnsolvedEqs : sublist( SolSet, lambda( [x], not EquationP( x ) ) ),

        /* and the solutions. */
        SolSet : sublist( SolSet, 'EquationP ),

        /* If no complete backsubstitution is requested then variables on the */
        /* right-hand sides of the solutions will only be substituted if they */
        /* do not belong to the variables specified in the command line.      */
        if not Solver_Backsubst then
          InternalSols : sublist(
            SolSet,
            lambda( [x], not member( lhs( x ), UserVars ) )
          ),

        /* Evaluate all variables and expressions with the solutions. */
        for Var in append( UserVars, Expressions ) do (

          EvalVar : if member( Var, map(lhs, SolSet) ) or not atom( Var ) then

            /* There may be errors when indeterminate expressions are */
            /* encountered.                                           */

            errcatch(
              if Solver_Backsubst then
                ev( Var, SolSet, infeval )
              else (
                TempVar : ev( Var, SolSet ),
                ev( TempVar, InternalSols, infeval )
              )
            )

          else
            [],

          if EvalVar # [] then
            UsrSolSet : endcons( Var = EvalVar[1], UsrSolSet )
          else
            PrintMsg( 'SHORT, SolverMsg["NoSol"], Var )
        ),


        /* append the unsolved equations. */
        if not Empty( UnsolvedEqs ) then
          UsrSolSet : endcons( UnsolvedEqs, UsrSolSet ),

        if Solver_RatSimp_Sols then
          UsrSolSet : errcatch( fullratsimp( UsrSolSet ) )
        else
          UsrSolSet : [UsrSolSet],

        if UsrSolSet = [] then
          PrintMsg( 'SHORT, SolverMsg["SolSetDrp"] )
        else
          UserSolutions : endcons( UsrSolSet[1], UserSolutions )

      ), /* END for SolSet */

      if Solver_Dump_To_File then
        DumpToFile( UserSolutions, [], [] )

    ), /* END if not Empty( Solutions ) */


    return( UserSolutions )

  ) /* END block */
)$


/******************************************************************************/
/* ExpandSolutionHierarchy transforms the hierarchically structured list of   */
/* solutions into a list of flat lists of solutions.                          */
/******************************************************************************/

ExpandSolutionHierarchy( Solutions ) := (

  mode_declare(
    Solutions, list
  ),

  block(
    [ FlatSolutions ],

    mode_declare(
      FlatSolutions, list
    ),

    if length( Solutions ) = 0 then return ( [] )
    /* listp = true indicates an additional recursion level */
    else if listp( last( Solutions ) ) then (

      FlatSolutions : rest( Solutions, -1 ),

      return(

        map(
          lambda( [ x ], append( FlatSolutions, x ) ),

          apply(
            'append,
            map( 'ExpandSolutionHierarchy, last( Solutions ) )
          )
        )

      )

    )

    else
      return( [ Solutions ] )

  )
)$


/******************************************************************************/
/* DumpToFile dumps the current set of solutions, equations and variables to  */
/* the file <Solver_Dump_File>.                                               */
/******************************************************************************/

DumpToFile( Sols, Eqs, Vars ) := (

  mode_declare(
    [ Sols, Eqs, Vars ], list
  ),

  block(
    [ Solutions, Equations, Variables ],

    PrintMsg( 'SHORT, SolverMsg["Dump"], Solver_Dump_File ),

    apply(
      'StringOut,
      [
        Solver_Dump_File,
        'Solutions = Sols,
        'Equations = Eqs,
        'Variables = Vars
      ]
    )
  )
)$

tma():=trace(
TerminateSolver,
SetupSolver,
ParamConsistency,
SolverAssumeZero,
ListOfPowers,
ImmediateAssignments,
LinearSolver,
MakeSquareLinearBlocks,
DelEqBeforeVar,
ComplCoeffMatrix,
LinCoeff,
ValuationSolver,
SolutionOK,
ValuationMatrix,
SetValuation,
Valuation,
OccurrenceMatrix,
Occurences,
MinVarPathsFirst,
PostProcess,
ExpandSolutionHierarchy,
DumpToFile)$

\end{verbatim}
