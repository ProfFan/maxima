Technische Dimensionierungsaufgaben erfordern h"aufig die L"osung von
Gleichungssystemen, die ein Objekt mathematisch beschreiben, nach den Werten
der funktionsbestimmenden Bauelementeparameter. Zur analytischen L"osung
kleinerer Dimensionierungsprobleme bietet sich der Einsatz kommerzieller
Computeralgebrasysteme wie Macsyma \cite{Macsyma} an, die in der Lage sind,
umfangreiche Gleichungen algebraisch zu manipulieren und nach ihren Variablen
symbolisch aufzul"osen.

Trotz ihrer hohen Leistungsf"ahigkeit sind diese Systeme jedoch meist
bereits "uberfordert, wenn lineare oder schwach nichtlineare, parametrisierte
Gleichungssysteme nach nur einer Teilmenge ihrer Variablen zu l"osen oder
zumindest symbolisch vorzuverarbeiten sind. Um solche typischerweise bei
Entwurfsaufgaben entstehenden Gleichungssystem behandeln zu k"onnen, wurde im
Rahmen dieser Arbeit ein auf heuristischen Algorithmen basierender,
universeller symbolischer Gleichungsl"oser entwickelt und in Macsyma
implementiert. Das Programmodul \verb+SOLVER+ erweitert die Funktionalit"at
der Macsyma-Befehle \verb+SOLVE+ und \verb+LINSOLVE+ zur symbolischen L"osung
algebraischer Gleichungen bzw. linearer Gleichungssysteme um die F"ahigkeit
zur selektiven L"osung nichtlinearer, parametrisierter Systeme mit
Freiheitsgraden.

Das erste Kapitel der vorliegenden Arbeit beschreibt einige
Anwendungsbereiche symbolischer Dimensionierungsmethoden, die aus ihnen
folgenden Anforderungen an einen symbolischen Gleichungsl"oser sowie die
verwendeten heuristischen Algorithmen zur Extraktion linearer Gleichungen und
zur Komplexit"atsbewertung algebraischer Funktionen. Das zweite Kapitel
enth"alt einen "Uberblick "uber die Struktur des {\em Solvers} und eine
Anleitung zu seiner Benutzung. Im Anhang findet sich der Quelltext des in der
internen h"oheren Programmiersprache von Macsyma implementierten Moduls
\verb+SOLVER.MAC+.
